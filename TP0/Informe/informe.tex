\documentclass[a4paper, 10pt]{article}
\usepackage[utf8]{inputenc}
\usepackage[spanish]{babel}
\usepackage{graphicx}
\usepackage{geometry}
\usepackage{listings}
\usepackage{amsmath}
\usepackage{amsfonts}
\usepackage{amssymb}
\usepackage{caratula}

\newcommand{\Z}{\mathbb{Z}}
\def\code#1{\texttt{#1}}
\newcommand\tab[1][0.5cm]{\hspace*{#1}}

\geometry{a4paper, margin=0.7in}

\begin{document}
    %Caratula
    \pagenumbering{gobble}
    \newpage

    \begin{center}
        \includegraphics[width=5cm, height=5cm]{images/logo}
    \end{center}

    \materia{Organización de Computadoras}
    \submateria{Segundo Cuatrimestre 2017}
    \titulo{Trabajo Práctico 0}

    \integrante{Rodrigo De Rosa}{97799}{rodrigoderosa@outlook.com}
    \integrante{Marcos Schapira}{97934}{schapiramarcos@gmail.com}
    \integrante{Facundo Guerrero}{97981}{facundoiguerrero@gmail.com}
    \maketitle
    %Fin caratula
    %Table of contents
    \newpage
    \pagenumbering{roman}
    \tableofcontents
    %Fin table of contents
    %Informe
    \newpage
	\pagenumbering{arabic}
	\section{Diseño e Implementación}
		En este trabajo práctico inicial, cuyo objetivo es el de familiarizarnos con el
		entorno de desarrollo que utilizaremos en el cuatrimestre, se implementa un programa
		que recibe una entrada de texto e identifica los palíndromos que se encuentran en
		ella.
		\subsection{Estructura del problema}
			La entrada de texto previamente mencionada es una cadena de caracteres \emph{ASCII}
			sin ninguna restricción. Dentro de esta cadena son consideradas \emph{palabras} aquellas
			que están compuestas por los caracteres:
			\begin{itemize}
				\item $a-z$
				\item $0-9$
				\item $\_$ y $-$
			\end{itemize}
			\tab Cualquier otro caracter \emph{ASCII} es considerado un \emph{espacio}. Es decir,
			indica el fin de una \emph{palabra} y el comienzo de otra. Cabe destacar que una cadena
			con un sólo caracter no es considerada \emph{palabra}.
		\subsection{Entorno}
			El trabajo se realizó en una máquina virtual \emph{NetBSD} (que simula tener un procesador
			\emph{MIPS}) montada por el emulador \emph{GXemul} en \emph{Ubuntu 17.04}.
		\subsection{Complicaciones}
			La principal complicación que surgió en el desarrollo del trabajo fue el hecho de que algunas
			librerías que existen en \emph{Ubuntu} no existen en \emph{NetBSD} (particularmente \emph{argp}),
			por lo que hubo que adaptarse a esta limitación y utilizar librerías que funcionaran en ambos. \\
			\tab Por otro lado, cierta secuencia de lineas de código funcionaban en un sistema operativo pero
			no el otro. Esta fue la mayor dificultad que se debió afrontar, pues si bien inicialmente el programa
			funcionaba, al probar en el otro sistema operativo se descubría que no era así. \\
			\tab Por último, un problema a resolver fue el de tener que enviar por \code{scp} todos los archivos
			que fueran modificados en \emph{Ubuntu} hacia \emph{NetBSD}. De todos modos este problema fue
			resuelto utilizando \code{sshfs} que permite utilizar la interfaz gráfica de \emph{Ubuntu} para
			modificar un directorio en la máquina virtual.
		\subsection{Desarrollo}
			El programa fue implementado en lenguaje C con sus librerías estándar y se utilizó la libreria
			\code{getopt} para facilitar el parseo de los flags. \\
			\tab En cuanto al problema en sí, la solución implementada consiste en buscar las previamente
			llamadas \emph{palabras} dentro de una cadena de caracteres y verificar si invertidas son iguales
			a su versión original (esto indica que son palíndromos). Para hacer esto se utilizó una implementación
			de la función \code{strrev} que, si bien en las versiones más actuales de C viene en las librerías 
			estándar, en \emph{NetBSD} no está dentro de estas.
	\section{Ejecución}
		\subsection{Instrucciones para la compilación}
			Para compilar el programa se debe abrir una consola en el directorio donde se encuentra el archivo
			fuente (\code{tp.c}) y correr el comando: \code{gcc -Wall tp.c [-o OUTPUT]}.
		\subsection{Instrucciones para la ejecución}
			Suponiendo que nuestro archivo ejecutable fuera \code{tp0}, los comandos de consola para ejecutarlo
			son:
			\begin{itemize}
				\item \code{./tp0 -h} para ver la ayuda.
				\item \code{./tp0 -v} para ver la versión.
				\item \code{./tp0 -i ~/INPUT -o ~/OUTPUT}  para correr el programa con \code{INPUT} como archivo de entrada
				y \code{OUTPUT} como archivo de salida. Ambos son opcionales y son reemplazados por \code{stdin} y \code{stdout}
				respectivamente.
			\end{itemize}
		\subsection{Pruebas}
			Para probar el correcto funcionamiento del programa se utilizaron tres archivos de prueba. A continuación
			se muestra la composición de dichos archivos y los resultados de las ejecuciones.
			\subsubsection{Primer prueba}
				\textbf{Entrada:}\\
				\tab\tab\code{Somos los primeros en completar el TP 0.}\\
				\tab\tab\code{Ojo que la fecha de entrega del TP0 es el martes 12 de Septiembre.}\\
				\tab\textbf{Salida:}\\
				\tab\tab\code{Somos}\\
				\tab\tab\code{Ojo}
			\subsubsection{Segunda prueba}
				\textbf{Entrada:}\\
				\tab\tab\code{MeNEm neUquEn 1a2d323d2a1 adke}\\
				\tab\tab\code{pepe$)$nene$/$larral$=$dom-mod?a23$\_$32a}\\
				\tab\textbf{Salida:}\\
				\tab\tab\code{MeNEm}\\
				\tab\tab\code{neUquEn}\\
				\tab\tab\code{1a2d323d2a1}\\
				\tab\tab\code{larral}\\
				\tab\tab\code{dom-mod}\\
				\tab\tab\code{a23$\_$32a}
			\subsubsection{Tercera prueba}
				\textbf{Entrada:}
				\begin{lstlisting}
    aD-2eT_R_Te2-Da%4004?CheVr
    peep23***   avion{daad}
    neUqUeN&NarNran
		        	\end{lstlisting}
				\tab\textbf{Salida:}
				\begin{lstlisting}
    aD-2eT_R_Te2-Da
    4004
    daad
    neUqUeN
    NarNran	
		        	\end{lstlisting}
\end{document}