\documentclass[a4paper, 10pt]{article}
\usepackage[utf8]{inputenc}
\usepackage[spanish]{babel}
\usepackage{graphicx}
\usepackage{geometry}
\usepackage{listings}
\usepackage{amsmath}
\usepackage{amsfonts}
\usepackage{amssymb}
\usepackage{caratula}

\newcommand{\Z}{\mathbb{Z}}
\def\code#1{\texttt{#1}}
\newcommand\tab[1][0.5cm]{\hspace*{#1}}

\geometry{a4paper, margin=0.7in}

\begin{document}
    %Caratula
    \pagenumbering{gobble}
    \newpage

    \begin{center}
        \includegraphics{images/logo}
    \end{center}

    \materia{Organización de Computadoras}
    \submateria{Segundo Cuatrimestre 2017}
    \titulo{Trabajo Práctico 1}

    \integrante{Rodrigo De Rosa}{97799}{rodrigoderosa@outlook.com}
    \integrante{Marcos Schapira}{97934}{schapiramarcos@gmail.com}
    \integrante{Facundo Guerrero}{97981}{facundoiguerrero@gmail.com}
    \maketitle
    %Fin caratula
    %Table of contents
    \newpage
    \pagenumbering{roman}
    \tableofcontents
    %Fin table of contents
    %Informe
    \newpage
	\pagenumbering{arabic}
	\section{Diseño e Implementación}
		En este trabajo práctico, cuyo objetivo es familiarizarse con el conjunto de 
		instrucciones MIPS y el concepto de ABI, 
		se implementa el mismo programa que el del TP0 pero esta vez utilizando assembly MIPS. 
		Dicho programa recibe una entrada de texto e identifica los palíndromos que se encuentran en
		ella.
		\subsection{Estructura del problema}
			La entrada de texto previamente mencionada es una cadena de caracteres \emph{ASCII}
			sin ninguna restricción. Dentro de esta cadena son consideradas \emph{palabras} aquellas
			que están compuestas por los caracteres:
			\begin{itemize}
				\item $a-z$
				\item $0-9$
				\item $\_$ y $-$
			\end{itemize}
			\tab Cualquier otro caracter \emph{ASCII} es considerado un \emph{espacio}. Es decir,
			indica el fin de una \emph{palabra} y el comienzo de otra. Cabe destacar que una cadena
			con un sólo caracter es considerada \emph{palabra}.
		\subsection{Entorno}
			El trabajo se realizó en una máquina virtual \emph{NetBSD} (que simula tener un procesador
			\emph{MIPS}) montada por el emulador \emph{GXemul} en \emph{Ubuntu 17.04}.
		\subsection{Complicaciones}
			
			Durante el desarrollo de el trabajo practico que esta siendo presentado, se presentaron muchas dificultades. \\
			La primera de ellas fue al momento de usar funciones que se encontraban en módulos externos, es decir, cuando se quería llamar desde un módulo main a una función auxiliar. El problema en si, fue que al llamar solamente a una función externa el salto funcionaba, pero cuando desde un main se querían hacer 2 saltos a funciones externas, dicho programa arrojaba un \code{Segmentation Fault}. Por otro lado, al intentar debuggear el programa con \code{GDB}, no se obtenía ningún tipo de información ya que de \code{GDB} se obtenía la siguiente salida:
			
		\begin{verbatim}
			Starting program: /root/TPs/TP1/tests/buffer/buffer

			Program received signal SIGSEGV, Segmentation fault.
			warning: Warning: GDB can't find the start of the function at 0x27bdffc8.

   				GDB is unable to find the start of the function at 0x27bdffc8
			and thus can't determine the size of that function's stack frame.
				This means that GDB may be unable to access that stack frame, or
			the frames below it.
    		This problem is most likely caused by an invalid program counter or
			stack pointer.
    			However, if you think GDB should simply search farther back
			from 0x27bdffc8 for code which looks like the beginning of a
			function, you can increase the range of the search using the `set
			heuristic-fence-post' command. 0x27bdffc8 in ?? ()
		\end{verbatim}
		
		Luego de continuar un arduo análisis del problema, y con ayuda del grupo de consultas, se pudo advertir que el error se encontraba al momento de salvar el \code{ra} al inicio de una función, y restaurarlo al final de la misma.
		\\
		Una vez superada dicha complicación, se pudo integrar casi toda la funcionalidad del programa que se tenía hasta el momento. Como se menciono recientemente, lo primero que se hizo fue programar funciones por separado que realizaban las distintas tareas necesarias para el correcto funcionamiento del programa, las cuales leían una entrada con un \code{SYS$\_$READ}, el cual era guardado en un buffer definido en la sección de \code{$.$data del $.$S}. En este caso, el parámetro del syscall pasado en a2 (available space) también estaba hardcodeado y era el mismo tamaño del buffer. Una vez logrado esto, el siguiente paso fue introducir el buffer dinámico, que fue donde apareció la siguiente complicación. Entonces, lo que se quería hacr ahora es que el buffer fuera dinámico, es decir que recibiera por parámetro su tamaño y se llamara a mymalloc para obtener el puntero a el mismo. En este caso, el programa terminaba con un \code{Segmentation Fault}. Al intentar debuggear el programa con \code{GDB}, se obtuvo una salida muy similar a la presentada anteriormente, por lo que tampoco se pudo obtener información relevante sobre el error. Luego de realizar el análisis pertinente, pudimos arreglar el error que se encontraba al momento de realizar el pedido de memoria con la función \code{my$\_$malloc}.
		\\
		También se puede notar, que una complicación adicional fue la poca información que se obtenía de \code{GDB} cuando se corrían los programas que terminaban con \code{Segmentation Fault}.



		\subsection{Desarrollo}
			El programa fue implementado en lenguaje \code{C} y \code{assembly MIPS}.
			\\
			Dicho programa se inicia en \code{C}. Aquí se realiza la validación y el procesamiento de los parámetros recibidos, la apertura de archivos y su respectivo manejo de errores. Luego de tener los archivos abiertos, se obtiene el \code{FILE DESCRIPTOR} de los mismos con la función \code{fileno} y se pasa el control a la función \code{palindrome}, la cual esta implementada en \code{MIPS}. Por ultimo, en \code{C} también se realiza el manejo de errores, si los hubiese, una vez que el la función \code{palindrome} devuelve el control al programa en \code{c}.
			\\
			Entonces una vez que la función \code{palindrome} tiene el control del programa, esta se encarga de llamar a las distintas funciones distribuidas en módulos que se encargan de identificar, procesar e imprimir los componentes léxicos que resulten ser palíndromos.
			La lógica que siguen dichas funciones es la que se presentara a continuación. Luego de almacenar los parámetros recibidos desde la función de \code{C}, la función \code{palindrome} reserva memoria para los buffer de entrada y salida a utilizar. Dicha operación, se lleva a cabo utilizando la función \code{my$\_$malloc} provista por la cátedra. Cabe aclarar, que también se realiza el manejo de errores en caso de ser necesario. Luego de tener el espacio de los buffers reservados, se llama a la función \code{get$\_$word} la cual se encarga de formar una palabra para que posteriormente sea procesada por \code{is$\_$palindrome}. La función \code{get$\_$word} llama a \code{get$\_$char} la cual se encarga de leer un \code{char} del buffer estático de entrada, y en el caso de que este vacío se encarga de volver a llenarlo. Luego de realizar dichas operaciones, se devuelve el carácter leído a \code{get$\_$word} la cual los acumula y verifica si dicho carácter recientemente leído es valido, o un espacio. En este ultimo caso, se escribe agrega un fin de linea al final de la palabra que esta próxima a procesarse. Contrariamente se vuelve a llamar a la función \code{get$\_$char} hasta completar una palabra.
			\\
			A continuación se detalla la documentación explicita de las funciones implementadas.
		
		\subsection{Manejo de errores}
			A lo largo del desarrollo del programa se definen ciertos errores para manejar posibles fallas del
			programa y así lograr un funcionamiento controlado y acorde. Estas son:
			\begin{itemize}

				\item \code{ALLOC$\_$ERROR}
				\\\textit{El error se puede dar al llamar a la función \code{malloc}.
				Junto a su mensaje específico se imprime a la vez el código generado por strerror en
				la anterior función.}
					\subitem \textbf{Mensaje:}
						\subsubitem \code{An error ocurred while allocating memory!}

				\item \code{REALLOC$\_$ERROR}
				\\\textit{El error se puede dar al llamar a la función \code{realloc}.
				Junto a su mensaje específico se imprime a la vez el código generado por strerror en
				la anterior función.}
					\subitem \textbf{Mensaje:}
						\subsubitem \code{An error ocurred while reallocating memory!}

				\item \code{INPUT$\_$OPEN$\_$ERROR}
				\\\textit{El error se puede dar al llamar la función \code{fopen}.
				Junto a su mensaje específico se imprime a la vez el código generado por strerror en
				la anterior función.}
					\subitem \textbf{Mensaje:}
						\subsubitem \code{An error ocurred while opening input file!}

				\item \code{OUTPUT$\_$OPEN$\_$ERROR}
				\\\textit{El error se puede dar al llamar la función \code{fopen}.
				Junto a su mensaje específico se imprime a la vez el código generado por strerror en
				la anterior función.}
					\subitem \textbf{Mensaje:}
						\subsubitem \code{An error ocurred while opening output file!}

				\item \code{RESULT$\_$WRITING$\_$ERROR}
				\\\textit{El error se puede dar al llamar la función \code{fprintf}
				si no se logró escribir todo el mensaje o si algo falló.
				Junto a su mensaje específico se imprime a la vez el código generado por strerror en
				la anterior función.}
					\subitem \textbf{Mensaje:}
						\subsubitem \code{An error ocurred while writing the result!}

				\item \code{PALINDROME$\_$ERROR$\_$MESSAGE}
				\\\textit{El error se puede dar al llamar a la función interna \code{get$\_$palindromes}.
				Esta devuelve \code{NULL} en caso de fallar (junto con su adecuado mensaje, explicado
				a continuación en el informe).}
					\subitem \textbf{Mensaje:}
						\subsubitem \code{An error ocurred while checking for palindromes!}

			\end{itemize}
		\subsubsection{Valores devueltos por la función \code{main}}
			Los siguientes códigos son mensajes devueltos por la función \code{main} al utilizar las funciones
			internas del programa(documentadas en la próxima sección del informe). Algunos de estos valores,
			en especial \code{FAIL} y \code{SUCCESS} son utilizados en otras funciones como valores booleanos
			False y True respectivamente.
			\begin{itemize}
				\item \code{SUCCESS} valor 0
					\\\textit{valor booleano de éxito.}
				\item \code{FAIL} valor 1
					\\\textit{valor booleano de falla.}
				\item \code{PALINDROME$\_$ERROR} valor mayor estricto a cero.
					\\\textit{ocurre cuando la función \code{palindrome} falla. 
					Esto puede deberse a los siguientes errores:}
					\subitem \code{INPUT$\_$MALLOC$\_$ERROR} valor 1.
					\\\textit\tab\tab{ocurre cuando la función \code{malloc} falla 
					en el contexto de Input.}
					\subitem \code{OUTPUT$\_$MALLOC$\_$ERROR} valor 2.
					\\\textit\tab\tab{ocurre cuando la función \code{malloc} falla 
					en el contexto de Output.}
					\subitem \code{WRITE$\_$ERROR} valor 3.
					\\\textit\tab\tab{ocurre cuando hay un error de escritura en assembly MIPS.}
					\subitem \code{READ$\_$ERROR} valor 4.
					\\\textit\tab\tab{ocurre cuando hay un error de lectura en assembly MIPS.}
				\item \code{BAD$\_$ARGUMENTS} valor 4
					\\\textit{ocurre cuando la función \code{process$\_$params} devuelve este mismo
					codigo al no poder procesar los parámetros correctamente.}
				\item \code{BAD$\_$INPUT$\_$PATH} valor 5
					\\\textit{ocurre cuando la función \code{open$\_$input} devuelve \code{FAIL}.}
				\item \code{BAD$\_$OUTPUT$\_$PATH} valor 6
					\\\textit{ocurre cuando la función \code{open$\_$output} devuelve \code{FAIL}.}
				\item \code{READING$\_$ERROR} valor 7
					\\\textit{ocurre cuando la función \code{read$\_$input} devuelve \code{FAIL o NULL}.}
			\end{itemize}
		\subsection{Documentación}
			Las siguientes funciones fueron implementadas con el objetivo de encontrar una solución al problema
			en cuestión.

			\subsubsection{Funciones en assembly C}	
			\begin{itemize}

				\item \code{FILE$*$ open$\_$input(char$*$ path)}
				\\\textit{Abre el input$\_$file y se devuelve su fp.
				Si el path es \code{NULL},
				se utiliza \code{DEFAULT$\_$INPUT} siendo en este caso stdin.}
					\subitem \textbf{Parámetros:}
						\subsubitem \code{path}: Dirección del archivo a abrir
					\subitem \textbf{Return:}
						\subsubitem File Pointer de input o \code{DEFAULT$\_$INPUT}
						en caso de no especificar un path.
					\subitem \textbf{Errores Posibles:}
						\subsubitem \code{INPUT$\_$OPEN$\_$ERROR}

				\item \code{FILE$*$ open$\_$output(char$*$ path)}
				\\\textit{Abre el output$\_$file y se devuelve su fp. Si el path es \code{NULL}
				se utiliza \code{DEFAULT$\_$OUTPUT} siendo en este caso stdout.}
					\subitem \textbf{Parámetros:}
						\subsubitem \code{path}: Dirección del archivo a abrir
					\subitem \textbf{Return:}
						\subsubitem File Pointer de output o \code{DEFAULT$\_$OUTPUT} en caso de
						no especificar un path.
					\subitem \textbf{Errores Posibles:}
						\subsubitem \code{OUTPUT$\_$OPEN$\_$ERROR}

				\item \code{void close$\_$files(FILE$*$ fp1, FILE$*$ fp2)}
				\\\textit{Cierra los dos archivos recibidos.}
					\subitem \textbf{Parámetros:}
						\subsubitem \code{fp2}: File Pointer de archivo a cerrar
						\subsubitem \code{fp1}: File Pointer de archivo a cerrar

				\item \code{void print$\_$help()}
				\\\textit{Imprime por consola información de los comandos y sobre el
				uso del programa.}

				\item \code{void print$\_$version()}
				\\\textit{Imprime por consola la version del programa y los integrantes del grupo.}

				\item \code{int process$\_$params(int argc, char$**$ argv,
				char$**$ input$\_$file, char$**$ output$\_$file)}
				\\\textit{Procesa los parámetros de entrada del programa y almacena
				los paths correspondientes en los parámetros de la función.}
					\subitem \textbf{Parámetros:}
						\subsubitem \code{argc}: Cantidad de argumentos del programa
						\subsubitem \code{argv}: Vector de argumentos del programa
						\subsubitem \code{input$\_$file}: Puntero al string que contiene el path
						del input
						\subsubitem \code{output$\_$file}: Puntero al string que contiene el path
						del output
					\subitem \textbf{Return:}
						\subsubitem \code{SUCCESS o BAD$\_$ARGUMENTS} , en el segundo caso este valor
						es verificado y manejado en la función \code{main}.
						
			\end{itemize}
			
			\subsubsection{Funciones en assembly MIPS}		
			\begin{itemize}
			
				\item \code{palindrome}
				\\\textit{Maneja el buffer tanto de lectura como de escritura, verificando si las palabras
				a analizar son o no palíndromas.}
					\subitem \textbf{Parámetros:}
						\subsubitem \code{input file descriptor}: Puntero al input file descriptor
						\subsubitem \code{input buffer size}: Tamaño del buffer
						\subsubitem \code{output file descriptor}: Puntero al output file descriptor
						\subsubitem \code{output buffer size}: Tamaño del buffer de salida
					\subitem \textbf{Return:}
						\subsubitem \code{SUCCESS o ERROR} , en el segundo caso este valor
						es verificado y manejado en la función \code{main}.
					\subitem \textbf{Errores Posibles:}
						\subsubitem \code{INPUT$\_$MALLOC$\_$ERROR, OUTPUT$\_$MALLOC$\_$ERROR, 
						WRITE$\_$ERROR, READ$\_$ERROR}	
				
				\item \code{mymalloc}
				\\\textit{Funcion malloc implementada en assembly MIPS.}
					\subitem \textbf{Parámetros:}
						\subsubitem \code{size}: Tamaño de memoria a pedir
					\subitem \textbf{Return:}
						\subsubitem \code{PUNTERO$\_$AL$\_$BLOQUE o ERROR} , en el segundo caso este valor
						es verificado y manejado en la función \code{palindrome}
				
				\item \code{myfree}
				\\\textit{Funcion free implementada en assembly MIPS.}
					\subitem \textbf{Parámetros:}
						\subsubitem \code{pointer}: Puntero al bloque a liberar.
					\subitem \textbf{Return:}
						\subsubitem \code{SUCCESS o ERROR}
						
				\item \code{myrealloc}
				\\\textit{Funcion realloc implementada en assembly MIPS.}
					\subitem \textbf{Parámetros:}
						\subsubitem \code{old$\_$pointer}: Puntero vector original
						\subsubitem \code{size}: Tamaño del vector original
						\subsubitem \code{size$\_$inc}: Incremento de memoria
					\subitem \textbf{Return:}
						\subsubitem \code{PUNTERO$\_$AL$\_$BLOQUE o ERROR}
						
				\item \code{get$\_$word}
				\\\textit{Lee una palabra, separando como espacios a los caracteres anteriormente mencionados.}
					\subitem \textbf{Parámetros:}
						\subsubitem \code{input file descriptor}: Puntero al input file descriptor
						\subsubitem \code{len$\_$pointer}: puntero donde se guarda el largo de la palabra leida
						\subsubitem \code{input buffer}: Puntero al buffer de entrada
					\subitem \textbf{Return:}
						\subsubitem \code{PALABRA o ERROR} , en el segundo caso este valor
						es verificado y manejado en la función \code{palindrome}
					\subitem \textbf{Errores Posibles:}
						\subsubitem \code{READ$\_$ERROR}
				
				\item \code{get$\_$char}
				\\\textit{Lee un caracter.}
					\subitem \textbf{Parámetros:}
						\subsubitem \code{input file descriptor}: Puntero al input file descriptor
						\subsubitem \code{input buffer}: Puntero al buffer de entrada
					\subitem \textbf{Return:}
						\subsubitem \code{CARACTER o ERROR} , en el segundo caso este valor
						es verificado y manejado en la función \code{get$\_$word}
					\subitem \textbf{Errores Posibles:}
						\subsubitem \code{READ$\_$ERROR}
						
				\item \code{is$\_$palindrome}
				\\\textit{Verifica si una palabra es o no palíndroma.}
					\subitem \textbf{Parámetros:}
						\subsubitem \code{palabra}: Palabra a analizar
						\subsubitem \code{size}: Longitud de la palabra
					\subitem \textbf{Return:}
						\subsubitem \code{VALOR BOOLEANO}
						
				\item \code{put$\_$char}
				\\\textit{escribe una palabra caracter.}
					\subitem \textbf{Parámetros:}
						\subsubitem \code{output file descriptor}: Puntero al output file descriptor
						\subsubitem \code{palindromo}: Palíndromo a escribir
						\subsubitem \code{output buffer}: Puntero al buffer de entrada
					\subitem \textbf{Return:}
						\subsubitem \code{SUCCESS o WRITE$\_$FAIL} , en el segundo caso este valor
						es verificado y manejado en la función \code{palindrome}
					\subitem \textbf{Errores Posibles:}
						\subsubitem \code{WRITE$\_$ERROR}																
			\end{itemize}
			
	\section{Ejecución}
		\subsection{Instrucciones para la compilación}
			Para compilar el programa se debe abrir una consola en el directorio donde se encuentra el archivo
			fuente (\code{tp.c}) y correr el comando: \code{gcc -Wall tp1.c [-o OUTPUT]}.
		\subsection{Instrucciones para la ejecución}
			Suponiendo que nuestro archivo ejecutable fuera \code{tp1}, los comandos de consola para ejecutarlo
			son:
			\begin{itemize}
				\item \code{./tp1 -h} para ver la ayuda.
				\item \code{./tp1 -v} para ver la versión.
				\item \code{./tp1 -i ~/INPUT -o ~/OUTPUT}  para correr el programa con \code{INPUT} como archivo de entrada
				y \code{OUTPUT} como archivo de salida. Ambos son opcionales y son reemplazados por \code{stdin} y \code{stdout}
				respectivamente.
			\end{itemize}
		\subsection{Pruebas}
			Para probar el correcto funcionamiento del programa se utilizo un set de prueba. A continuación
			se muestra la composición y resultados de las ejecuciones de dicho set. Además, notar que
      desde la prueba 1 hasta la prueba 9, la entrada es mediante un archivo, es decir que
      se provee el archivo mediante \code{-i test$\_$inputN} ( con N = número de prueba), y la salida
      por terminal. Por último, vale aclarar que las pruebas se realizaron tanto en \code{Ubuntu}
      como en \code{NetBSD}.

      \subsubsection{Prueba 1}
        Caso de prueba provisto por la cátedra. Vale aclarar que un carácter es considerado
        palíndromo.\\

				\textbf{Entrada:}\\
				\tab\tab\code{Somos los primeros en completar el TP 0.}\\
				\tab\tab\code{Ojo que la fecha de entrega del TP0 es el martes 12 de Septiembre.}\\

        \textbf{Salida:}\\
				\tab\tab\code{Somos}\\
        \tab\tab\code{0}\\
				\tab\tab\code{Ojo}

			\subsubsection{Prueba 2}
        Este caso de prueba intenta demostrar el correcto funcionamiento de la
        detección de espacios. Como se puede ver, en la primera linea las palabras
        están separadas por el carácter espacio, pero en la segunda se linea se
        intenta demostrar que los caracteres no validos ( ver sección 1.1 )
        también funcionan como espacios. Además, vale notar que la palabra
        palíndroma se detecta sin importar mayúsculas o minúsculas.\\

				\textbf{Entrada:}\\
				\tab\tab\code{MeNEm neUquEn 1a2d323d2a1 adke}\\
				\tab\tab\code{pepe$)$nene$/$larral$=$dom-mod?a23$\_$32a}\\

        \textbf{Salida:}\\
				\tab\tab\code{MeNEm}\\
				\tab\tab\code{neUquEn}\\
				\tab\tab\code{1a2d323d2a1}\\
				\tab\tab\code{larral}\\
				\tab\tab\code{dom-mod}\\
				\tab\tab\code{a23$\_$32a}

			\subsubsection{Prueba 3}
        El objetivo de esta prueba es ver el funcionamiento de los caracteres
        validos que no sean letras ni números. Como se vio en la sección 1.1,
        \code{$\-$} y \code{$\_$} deben ser considerados como caracteres validos.
        Entonces esta prueba quiere demostrar el funcionamiento de dichos
        caracteres. Ademas, nuevamente notar que la detección de las palabras
        no es \code{sensitive}.\\

				\textbf{Entrada:}\\
        \tab\tab\code{aD$\-$2eT$\_$R$\_$Te2$\-$Da$/$4004$?$CheVr}\\
        \tab\tab\code{peep23$***$   avion${daad}$}\\
        \tab\tab\code{neUqUeN$\&$NarNran}\\

        \textbf{Salida:}\\
        \tab\tab\code{aD$\-$2eT$\_$R$\_$Te2$\-$Da}\\
        \tab\tab\code{4004}\\
        \tab\tab\code{daad}\\
        \tab\tab\code{neUqUeN}\\
        \tab\tab\code{NarNran}\\

      \subsubsection{Prueba 4}
        Esta prueba tiene el objetivo de corroborar el caso borde donde la
        entrada es vaciá.\\

        \textbf{Entrada:}\\
        \tab\tab\code{ }\\

        \textbf{Salida:}\\
        \tab\tab\code{ }\\

      \subsubsection{Prueba 5}
        El objetivo de esta prueba es verificar el correcto funcionamiento de
        la detección un único carácter.\\

				\textbf{Entrada:}\\
        \tab\tab\code{a}\\

        \textbf{Salida:}\\
        \tab\tab\code{a}\\


      \subsubsection{Prueba 6}
        El objetivo de esta prueba es corroborar el correcto funcionamiento del
        programa cuando se alcanza el limite inicial de tamaño del string que
        contiene las palabras palíndromas. En esta prueba, se ingresa una palabra
        palíndroma de 128 caracteres que es la capacidad máxima inicial.\\
        \textbf{La entrada para las siguientes pruebas es continua. Se hace un salto
        de linea con fines ilustrativos.}\\

        \textbf{Entrada:}\\
        \tab\tab\code{aaaaaaaaaaaaaaaaaaaaaaaaaaaaaaaaaaaaaaaaaaaaaaaaaaaaaaaaaaaaaaaa-\\-aaaaaaaaaaaaaaaaaaaaaaaaaaaaaaaaaaaaaaaaaaaaaaaaaaaaaaaaaaaaaaaa}\\

        \textbf{Salida:}\\
        \tab\tab\code{aaaaaaaaaaaaaaaaaaaaaaaaaaaaaaaaaaaaaaaaaaaaaaaaaaaaaaaaaaaaaaaa-\\-aaaaaaaaaaaaaaaaaaaaaaaaaaaaaaaaaaaaaaaaaaaaaaaaaaaaaaaaaaaaaaaa}\\

      \subsubsection{Prueba 7}
        Esta prueba intenta demostrar el funcionamiento adecuado del
        programa cuando se supera el tamaño inicial del string donde se guardan
        las palabras palíndromas. Aquí, se ingresa una palabra palíndroma de 129
        caracteres, es decir que se supera en 1 carácter la capacidad máxima.
        De esta manera, se esta forzando al programa a aumentar la capacidad
        máxima de almacenado. En otras palabras, se intenta probar el correcto
        funcionamiento del caso borde de la memoria, donde el programa debe
        hacer un realloc. Notar que se presentan dos casos de prueba, uno donde
        una misma palabra supera el limite, y en segundo lugar con palabras
        distintas.\\
        \textbf{La entrada para las siguientes pruebas es continua. Se hace un salto
        de linea con fines ilustrativos.}\\

        \textbf{Entrada 1:}\\
        \tab\tab\code{aaaaaaaaaaaaaaaaaaaaaaaaaaaaaaaaaaaaaaaaaaaaaaaaaaaaaaaaaaaaaaa-\\-aaaaaaaaaaaaaaaaaaaaaaaaaaaaaaaaaaaaaaaaaaaaaaaaaaaaaaaaaaaaaaaaaa}\\

        \textbf{Salida 1:}\\
        \tab\tab\code{aaaaaaaaaaaaaaaaaaaaaaaaaaaaaaaaaaaaaaaaaaaaaaaaaaaaaaaaaaaaaaa-\\-aaaaaaaaaaaaaaaaaaaaaaaaaaaaaaaaaaaaaaaaaaaaaaaaaaaaaaaaaaaaaaaaaa}\\

        \textbf{Entrada 2:}\\
        \tab\tab\code{aaaaaaaaaaaaaaaaaaaaaaaaaaaaaaaaaaaaaaaaaaaaaaaaaaaaaaaaaaaaaaa-\\-aaaaaaaaaaaaaaaaaaaaaaaaaaaaaaaaaaaaaaaaaaaaaaaaaaaaaaaaaaaa menem}\\

        \textbf{Salida 2:}\\
        \tab\tab\code{aaaaaaaaaaaaaaaaaaaaaaaaaaaaaaaaaaaaaaaaaaaaaaaaaaaaaaaaaaaaaaa-\\-aaaaaaaaaaaaaaaaaaaaaaaaaaaaaaaaaaaaaaaaaaaaaaaaaaaaaaaaaaaa}\\
        \tab\tab\code{menem}\\

      \subsubsection{Prueba 8}
        Esta prueba tiene como objetivo verificar el funcionamiento del programa
        cuando se alcanza el máximo inicial lectura. Entonces, se provee al
        programa una entrada de 256 caracteres.\\
        \textbf{La entrada para las siguientes pruebas es continua. Se hace un salto
        de linea con fines ilustrativos.}\\

        \textbf{Entrada:}\\
        \tab\tab\code{aaaaaaaaaaaaaaaaaaaaaaaaaaaaaaaaaaaaaaaaaaaaaaaaaaaaaaaaaaaaaaaa-\\-aaaaaaaaaaaaaaaaaaaaaaaaaaaaaaaaaaaaaaaaaaaaaaaaaaaaaaaaaaaaaaaa-\\-aaaaaaaaaaaaaaaaaaaaaaaaaaaaaaaaaaaaaaaaaaaaaaaaaaaaaaaaaaaaaaaa-\\-aaaaaaaaaaaaaaaaaaaaaaaaaaaaaaaaaaaaaaaaaaaaaaaaaaaaaaaaaaaaaaaa}\\

        \textbf{Salida:}\\
        \tab\tab\code{aaaaaaaaaaaaaaaaaaaaaaaaaaaaaaaaaaaaaaaaaaaaaaaaaaaaaaaaaaaaaaaa-\\-aaaaaaaaaaaaaaaaaaaaaaaaaaaaaaaaaaaaaaaaaaaaaaaaaaaaaaaaaaaaaaaa-\\-aaaaaaaaaaaaaaaaaaaaaaaaaaaaaaaaaaaaaaaaaaaaaaaaaaaaaaaaaaaaaaaa-\\-aaaaaaaaaaaaaaaaaaaaaaaaaaaaaaaaaaaaaaaaaaaaaaaaaaaaaaaaaaaaaaaa}\\

      \subsubsection{Prueba 9}
        El objetivo de esta prueba es corroborar el correcto funcionamiento del
        programa cuando se supera el tamaño inicial de lectura. Con dicho motivo,
        se ingresa una entrada de 257 caracteres, superando en 1 carácter la
        capacidad máxima. Al igual que la prueba 7, se esta probando el realloc
        pero en este caso para la lectura.
        \textbf{La entrada para las siguientes pruebas es continua. Se hace un salto
        de linea con fines ilustrativos.}\\

        \textbf{Entrada:}\\
        \tab\tab\code{aaaaaaaaaaaaaaaaaaaaaaaaaaaaaaaaaaaaaaaaaaaaaaaaaaaaaaaaaaaaaaaa-\\-aaaaaaaaaaaaaaaaaaaaaaaaaaaaaaaaaaaaaaaaaaaaaaaaaaaaaaaaaaaaaaaa-\\-aaaaaaaaaaaaaaaaaaaaaaaaaaaaaaaaaaaaaaaaaaaaaaaaaaaaaaaaaaaaaaaa-\\-aaaaaaaaaaaaaaaaaaaaaaaaaaaaaaaaaaaaaaaaaaaaaaaaaaaaaaaaaaaaaaaaa}\\

        \textbf{Salida:}\\
        \tab\tab\code{aaaaaaaaaaaaaaaaaaaaaaaaaaaaaaaaaaaaaaaaaaaaaaaaaaaaaaaaaaaaaaaa-\\-aaaaaaaaaaaaaaaaaaaaaaaaaaaaaaaaaaaaaaaaaaaaaaaaaaaaaaaaaaaaaaaa-\\-aaaaaaaaaaaaaaaaaaaaaaaaaaaaaaaaaaaaaaaaaaaaaaaaaaaaaaaaaaaaaaaa-\\-aaaaaaaaaaaaaaaaaaaaaaaaaaaaaaaaaaaaaaaaaaaaaaaaaaaaaaaaaaaaaaaaa}\\

      \subsubsection{Prueba 10}
        Se repite la prueba 1, esta vez con el objetivo de probar la entrada-salida vía archivos.
        La entrada se pasa por parámetro como  \code{-i test$\_$input1} y se pasa el archivo de salida
         \code{-i test$\_$output10}. La salida mostrada, es lo que contiene el archivo de salida
        anteriormente mencionado.\\

				\textbf{Entrada:}\\
				\tab\tab\code{Somos los primeros en completar el TP 0.}\\
				\tab\tab\code{Ojo que la fecha de entrega del TP0 es el martes 12 de Septiembre.}\\

        \textbf{Salida:}\\
				\tab\tab\code{Somos}\\
        \tab\tab\code{0}\\
				\tab\tab\code{Ojo}

      \subsubsection{Prueba 11}
        Se repite la prueba 2, esta vez con el objetivo de probar la entrada-salida vía terminal.
        Se correrá el programa sin ningún parámetro, y se hará el ingreso por entrada estándar
        (indicando su finalización con \code{Ctrl+D}) y se obtendrá la salida de la misma forma.\\

        \textbf{Entrada:}\\
        \tab\tab\code{MeNEm neUquEn 1a2d323d2a1 adke}\\
        \tab\tab\code{pepe$)$nene$/$larral$=$dom-mod?a23$\_$32a}\\

        \textbf{Salida:}\\
        \tab\tab\code{MeNEm}\\
        \tab\tab\code{neUquEn}\\
        \tab\tab\code{1a2d323d2a1}\\
        \tab\tab\code{larral}\\
        \tab\tab\code{dom-mod}\\
        \tab\tab\code{a23$\_$32a}

      \subsubsection{Prueba 12}
        Se repite la prueba 3, con el objetivo de verificar la entrada estándar y salida vía archivos.
        La salida se pasa por parámetros como \code{-i test$\_$input12}. La salida mostrada
        a continuación, es lo que contiene el archivo de salida anteriormente mencionado.\\

        \textbf{Entrada:}\\
        \tab\tab\code{aD$\-$2eT$\_$R$\_$Te2$\-$Da$/$4004$?$CheVr}\\
        \tab\tab\code{peep23$***$   avion${daad}$}\\
        \tab\tab\code{neUqUeN$\&$NarNran}\\

        \textbf{Salida:}\\
        \tab\tab\code{aD$\-$2eT$\_$R$\_$Te2$\-$Da}\\
        \tab\tab\code{4004}\\
        \tab\tab\code{daad}\\
        \tab\tab\code{neUqUeN}\\
        \tab\tab\code{NarNran}\\

	\section{Conclusiones}
		La principal conclusión que obtuvimos a partir del desarrollo de este trabajo práctico es que, al trabajar en \emph{NetBSD}
		se debe ser más cuidadoso al programar que en el caso de, por ejemplo, \emph{Ubuntu}. Con esto nos referimos a ciertas
		situaciones particulares en las que \emph{Ubuntu} resuelve ciertos problemas de la programación que \emph{NetBSD} no resuelve
		y resulta en un error. \\
		\tab Por ejemplo, en la línea 291 de \code{tp.c} se agrega un fin de linea que no es necesario en \emph{Ubuntu}, pues lo
		'resuelve solo', pero sí lo es en \emph{NetBSD} pues en este último, el programa no encuentra en fin de linea e imprime
		caracteres de más. \\
		\tab Por esta misma razón, notamos que es recomendable trabajar constantemente utilizando la carpeta virtual de \code{sshfs}
		para poder compilar y ejecutar el programa en \emph{NetBSD} en lugar de hacerlo en \emph{Ubuntu}. De todas maneras, al tener
		que desarrollar en \emph{Assembler}, no queda otra opción que hacer esto recién mencionado.
\end{document}
